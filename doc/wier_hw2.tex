% ============================================================================================
% This is a LaTeX template used for the course
%
%  I M A G E   B A S E D   B I O M E T R I C S
%
% Faculty of Computer and Information Science
% University of Ljubljana
% Slovenia, EU
%
% You can use this template for whatever reason you like.
% If you have any questions feel free to contact
% ziga.emersic@fri.uni-lj.si
% ============================================================================================

\documentclass[9pt]{IEEEtran}

% basic
\usepackage[english]{babel}
\usepackage{graphicx,epstopdf,fancyhdr,amsmath,amsthm,amssymb,url,array,textcomp,svg,listings,hyperref,xcolor,colortbl,float,gensymb,longtable,supertabular,multicol,placeins}

 % `sumniki' in names
\usepackage[utf8x]{inputenc}

 % search and copy for `sumniki'
\usepackage[T1]{fontenc}
\usepackage{lmodern}
\input{glyphtounicode}
\pdfgentounicode=1

% tidy figures
\graphicspath{{./figures/}}
\DeclareGraphicsExtensions{.pdf,.png,.jpg,.eps}

% correct bad hyphenation here
\hyphenation{op-tical net-works semi-conduc-tor trig-gs}

% ============================================================================================

\title{\vspace{0ex} %
% TITLE IN HERE:
Structured data extraction from the Web
\\ \normalsize{Web Information Extraction and Retrieval 2018/19, Faculty of Computer and Information Science, University of Ljubljana}}
\author{ %
% AUTHOR IN HERE:
Matej Klemen, Andraž Povše, Jaka Stavanja
\vspace{-4.0ex}
}

% ============================================================================================

\begin{document}

\maketitle

\begin{abstract}
In this work we present our implementation of 2 different approaches for structured data extraction from the Web: using regular expressions and using XPath.
We test the implemented methods on 6 webpages from 3 different sources (Overstock, Rtvslo and Avto.net) and provide the outputs for the pages, generated by the methods.
\end{abstract}

\section{Introduction}

The Web is an ever-increasing collection of information. 
Probably the most used format for representing this information is HTML.
After crawling a webpage (which was done in the previous assignment) we would like to be able to automatically extract useful parts of the page.
Unfortunately, the popularity of HTML makes this task a little harder, since its primary goal is to make pages readable by humans, not necessarily computers.

For this assignment, we implement 2 different approaches to structured data extraction for 6 webpages from 3 different sources (Overstock, Rtvslo and Avto.net). 
These are data extraction using regular expressions and using XPath query language.

The rest of this report is structured as follows. 
In chapter \ref{section:used-data} we present the chosen 2 additional webpages, on which we test our methods.
In chapter \ref{section:methodology} we present the implementations of the 2 methods for structured data extraction and mention the obtained results.
In chapter \ref{section:conclusion} we conclude with a summary of what was done in this assignment.

\section{Used data}
\label{section:used-data}

In addition to the provided webpages from \textit{Overstock} and \textit{Rtvslo}, we select another source from which we obtain 2 similar webpages.
The third source on which we test our implemented methods are 2 webpages from \textit{Avto.net}. The data items and data records we are interested in are shown in Figure \textbf{[TODO: ref image once its done]}.

\textbf{[TODO: describe the 2 selected webpages (from Avto.net) and show a picture with identification of data items and data records.]}

\section{Methodology}
\label{section:methodology}

In this section we describe our implementations of the approaches using regular expressions and XPath query language.

\subsection{Approach using regular expressions}
\label{section:regex}
\textbf{[TODO: describe implementation and reference the Appendix, where you put the output of regex method for the webpages]}

\subsection{Approach using XPath}
\label{section:xpath}
\textbf{[TODO: describe implementation and reference the Appendix, where you put the output of xpath method for the webpages]}

\begin{figure}[h]
    \centering
    \includegraphics[width=1\columnwidth]{plot1.jpg}
    \caption{A nice plot showing something really cool and awesome.}
    \label{fig:plot1}
\end{figure}

\section{Conclusion}
\label{section:conclusion}

We presented 2 approaches to structured data extraction from the Web: using regular expressions and using XPath query language.
We applied these methods to 6 webpages and provided the methods' output.
The third, more general, approach to structured data extraction (using RoadRunner algorithm) was not implemented.

\bibliographystyle{IEEEtran}
\bibliography{bibliography}

\appendix[Outputs of the methods]
\textbf{[TODO: provide outputs for all the webpages for each method (probably in the form of some dank lstlisting?]}
\end{document}